% visualtoks.tex, version 1.0a

% Copyright (C) 2025 plante
% This package is relased under the LaTeX Project Public License (LPPL) 1.3c.

% visualtoks: typeset TeXbook-style visualisations of token lists.

% Usage: `\visualtoks { <tokens> }`
% This package may be used in plain TeX or LaTeX by `% visualtoks.tex, version 1.0a

% Copyright (C) 2025 plante
% This package is relased under the LaTeX Project Public License (LPPL) 1.3c.

% visualtoks: typeset TeXbook-style visualisations of token lists.

% Usage: `\visualtoks { <tokens> }`
% This package may be used in plain TeX or LaTeX by `% visualtoks.tex, version 1.0a

% Copyright (C) 2025 plante
% This package is relased under the LaTeX Project Public License (LPPL) 1.3c.

% visualtoks: typeset TeXbook-style visualisations of token lists.

% Usage: `\visualtoks { <tokens> }`
% This package may be used in plain TeX or LaTeX by `% visualtoks.tex, version 1.0a

% Copyright (C) 2025 plante
% This package is relased under the LaTeX Project Public License (LPPL) 1.3c.

% visualtoks: typeset TeXbook-style visualisations of token lists.

% Usage: `\visualtoks { <tokens> }`
% This package may be used in plain TeX or LaTeX by `\input{visualtoks}`.

% See the documentation for typeset examples and caveats.



\ifdefined\visualtoks
	\expandafter\endinput
\fi
\edef\next{\catcode`\noexpand\@=\the\catcode`\@\space}
\catcode`\@=11

% \visualtoks@cycle { tokens }
% Apply \visualtoksA to each token in tokens. Space tokens, left and right braces
% are replaced with \visualtoks@cycle@space, \visualtoks@cycle@bgroup, and
% \visualtoks@cycle@egroup respectively. Expandable.
\long\def\visualtoks@cycle#1{\visualtoks@cycleA#1{\visualtoks@cycle@nil}\visualtoks@cycleF}
\long\def\visualtoks@cycleA#1#{%
	\expandafter\visualtoks@cycleD
		\romannumeral\visualtoks@cycleB{}#1 \visualtoks@cycle@nil
	\visualtoks@cycleF \visualtoks@cycleE
}
\long\def\visualtoks@cycleB#1#2 #3\visualtoks@cycle@nil{%
	\if\relax\detokenize{#3}\relax \visualtoks@cycleC{#1#2}\fi
	\visualtoks@cycleB#1#2\visualtoks@cycle@space#3\visualtoks@cycle@nil
}
\long\def\visualtoks@cycleC#1\fi#2\visualtoks@cycle@nil{\fi\z@ #1\visualtoks@cycle@nil}
\long\def\visualtoks@cycleD#1{%
	\ifx\visualtoks@cycle@nil#1\visualtoks@cycleF\fi
	\visualtoksA{#1}\visualtoks@cycleD
}
\long\def\visualtoks@cycleE#1{%
	\ifx\visualtoks@cycle@nil#1\visualtoks@cycleF\fi
	\visualtoksA\visualtoks@cycle@bgroup
	\visualtoks@cycle{#1}%
	\visualtoksA\visualtoks@cycle@egroup
	\visualtoks@cycleA
}
\long\def\visualtoks@cycleF#1\visualtoks@cycleF{\fi}

\protected\def\visualtoks@cycle@bgroup{\noexpand\visualtoks@cycle@bgroup}
\protected\def\visualtoks@cycle@egroup{\noexpand\visualtoks@cycle@egroup}
\protected\def\visualtoks@cycle@space {\noexpand\visualtoks@cycle@space}

\protected\def\visualtoks@cycle@nil{\noexpand\visualtoks@cycle@nil}

\long\def\visualtoks@firstoftwo#1#2{#1}
\long\def\visualtoks@secondoftwo#1#2{#2}

\newdimen\visualtokskip \visualtokskip=1em

\long\def\visualtoks#1{%
	\begingroup
		\leavevmode
		\ifcsname ttfamily\endcsname \ttfamily \else \tt \fi
		\raggedright\frenchspacing
		\hskip-\visualtokskip \visualtoks@cycle{#1}%
	\endgroup
}
\long\def\visualtoksA#1{%
	\hskip\visualtokskip
	\ifx\visualtoks@cycle@space#1%
		\char`\ $_{10}$%
	\else\ifx\visualtoks@cycle@bgroup#1%
		\char`\{$_1$%
	\else\ifx\visualtoks@cycle@egroup#1%
		\char`\}$_2$%
	\else
		\escapechar`\\%
		\edef\tempa{\string#1}%
		\escapechar\m@ne
		\edef\tempb{\string#1}%
		% #1 may be an \if... token, so we must use first/secondoftwo
		\ifx\tempa\tempb \expandafter\visualtoks@firstoftwo \else \expandafter\visualtoks@secondoftwo \fi
			{\string#1$_{%
				\ifcat $#13\fi
				\ifcat &#14\fi
				\ifcat###16\fi
				\ifcat ^#17\fi
				\ifcat _#18\fi
				\ifcat\space#110\fi
				\ifcat a#111\fi
				\ifcat !#112\fi
			}$}%
			{\visualtoksB{\string#1}}%
	\fi\fi\fi
}

\long\def\visualtoksB#1{%
	\hbox{%
		\vrule
		\vtop{%
			\vbox{%
				\hrule \kern\p@
				\hbox{\vphantom/\thinspace#1\thinspace}%
			}%
			\kern\p@ \hrule
		}%
		\vrule
	}%
}

\next
`.

% See the documentation for typeset examples and caveats.



\ifdefined\visualtoks
	\expandafter\endinput
\fi
\edef\next{\catcode`\noexpand\@=\the\catcode`\@\space}
\catcode`\@=11

% \visualtoks@cycle { tokens }
% Apply \visualtoksA to each token in tokens. Space tokens, left and right braces
% are replaced with \visualtoks@cycle@space, \visualtoks@cycle@bgroup, and
% \visualtoks@cycle@egroup respectively. Expandable.
\long\def\visualtoks@cycle#1{\visualtoks@cycleA#1{\visualtoks@cycle@nil}\visualtoks@cycleF}
\long\def\visualtoks@cycleA#1#{%
	\expandafter\visualtoks@cycleD
		\romannumeral\visualtoks@cycleB{}#1 \visualtoks@cycle@nil
	\visualtoks@cycleF \visualtoks@cycleE
}
\long\def\visualtoks@cycleB#1#2 #3\visualtoks@cycle@nil{%
	\if\relax\detokenize{#3}\relax \visualtoks@cycleC{#1#2}\fi
	\visualtoks@cycleB#1#2\visualtoks@cycle@space#3\visualtoks@cycle@nil
}
\long\def\visualtoks@cycleC#1\fi#2\visualtoks@cycle@nil{\fi\z@ #1\visualtoks@cycle@nil}
\long\def\visualtoks@cycleD#1{%
	\ifx\visualtoks@cycle@nil#1\visualtoks@cycleF\fi
	\visualtoksA{#1}\visualtoks@cycleD
}
\long\def\visualtoks@cycleE#1{%
	\ifx\visualtoks@cycle@nil#1\visualtoks@cycleF\fi
	\visualtoksA\visualtoks@cycle@bgroup
	\visualtoks@cycle{#1}%
	\visualtoksA\visualtoks@cycle@egroup
	\visualtoks@cycleA
}
\long\def\visualtoks@cycleF#1\visualtoks@cycleF{\fi}

\protected\def\visualtoks@cycle@bgroup{\noexpand\visualtoks@cycle@bgroup}
\protected\def\visualtoks@cycle@egroup{\noexpand\visualtoks@cycle@egroup}
\protected\def\visualtoks@cycle@space {\noexpand\visualtoks@cycle@space}

\protected\def\visualtoks@cycle@nil{\noexpand\visualtoks@cycle@nil}

\long\def\visualtoks@firstoftwo#1#2{#1}
\long\def\visualtoks@secondoftwo#1#2{#2}

\newdimen\visualtokskip \visualtokskip=1em

\long\def\visualtoks#1{%
	\begingroup
		\leavevmode
		\ifcsname ttfamily\endcsname \ttfamily \else \tt \fi
		\raggedright\frenchspacing
		\hskip-\visualtokskip \visualtoks@cycle{#1}%
	\endgroup
}
\long\def\visualtoksA#1{%
	\hskip\visualtokskip
	\ifx\visualtoks@cycle@space#1%
		\char`\ $_{10}$%
	\else\ifx\visualtoks@cycle@bgroup#1%
		\char`\{$_1$%
	\else\ifx\visualtoks@cycle@egroup#1%
		\char`\}$_2$%
	\else
		\escapechar`\\%
		\edef\tempa{\string#1}%
		\escapechar\m@ne
		\edef\tempb{\string#1}%
		% #1 may be an \if... token, so we must use first/secondoftwo
		\ifx\tempa\tempb \expandafter\visualtoks@firstoftwo \else \expandafter\visualtoks@secondoftwo \fi
			{\string#1$_{%
				\ifcat $#13\fi
				\ifcat &#14\fi
				\ifcat###16\fi
				\ifcat ^#17\fi
				\ifcat _#18\fi
				\ifcat\space#110\fi
				\ifcat a#111\fi
				\ifcat !#112\fi
			}$}%
			{\visualtoksB{\string#1}}%
	\fi\fi\fi
}

\long\def\visualtoksB#1{%
	\hbox{%
		\vrule
		\vtop{%
			\vbox{%
				\hrule \kern\p@
				\hbox{\vphantom/\thinspace#1\thinspace}%
			}%
			\kern\p@ \hrule
		}%
		\vrule
	}%
}

\next
`.

% See the documentation for typeset examples and caveats.



\ifdefined\visualtoks
	\expandafter\endinput
\fi
\edef\next{\catcode`\noexpand\@=\the\catcode`\@\space}
\catcode`\@=11

% \visualtoks@cycle { tokens }
% Apply \visualtoksA to each token in tokens. Space tokens, left and right braces
% are replaced with \visualtoks@cycle@space, \visualtoks@cycle@bgroup, and
% \visualtoks@cycle@egroup respectively. Expandable.
\long\def\visualtoks@cycle#1{\visualtoks@cycleA#1{\visualtoks@cycle@nil}\visualtoks@cycleF}
\long\def\visualtoks@cycleA#1#{%
	\expandafter\visualtoks@cycleD
		\romannumeral\visualtoks@cycleB{}#1 \visualtoks@cycle@nil
	\visualtoks@cycleF \visualtoks@cycleE
}
\long\def\visualtoks@cycleB#1#2 #3\visualtoks@cycle@nil{%
	\if\relax\detokenize{#3}\relax \visualtoks@cycleC{#1#2}\fi
	\visualtoks@cycleB#1#2\visualtoks@cycle@space#3\visualtoks@cycle@nil
}
\long\def\visualtoks@cycleC#1\fi#2\visualtoks@cycle@nil{\fi\z@ #1\visualtoks@cycle@nil}
\long\def\visualtoks@cycleD#1{%
	\ifx\visualtoks@cycle@nil#1\visualtoks@cycleF\fi
	\visualtoksA{#1}\visualtoks@cycleD
}
\long\def\visualtoks@cycleE#1{%
	\ifx\visualtoks@cycle@nil#1\visualtoks@cycleF\fi
	\visualtoksA\visualtoks@cycle@bgroup
	\visualtoks@cycle{#1}%
	\visualtoksA\visualtoks@cycle@egroup
	\visualtoks@cycleA
}
\long\def\visualtoks@cycleF#1\visualtoks@cycleF{\fi}

\protected\def\visualtoks@cycle@bgroup{\noexpand\visualtoks@cycle@bgroup}
\protected\def\visualtoks@cycle@egroup{\noexpand\visualtoks@cycle@egroup}
\protected\def\visualtoks@cycle@space {\noexpand\visualtoks@cycle@space}

\protected\def\visualtoks@cycle@nil{\noexpand\visualtoks@cycle@nil}

\long\def\visualtoks@firstoftwo#1#2{#1}
\long\def\visualtoks@secondoftwo#1#2{#2}

\newdimen\visualtokskip \visualtokskip=1em

\long\def\visualtoks#1{%
	\begingroup
		\leavevmode
		\ifcsname ttfamily\endcsname \ttfamily \else \tt \fi
		\raggedright\frenchspacing
		\hskip-\visualtokskip \visualtoks@cycle{#1}%
	\endgroup
}
\long\def\visualtoksA#1{%
	\hskip\visualtokskip
	\ifx\visualtoks@cycle@space#1%
		\char`\ $_{10}$%
	\else\ifx\visualtoks@cycle@bgroup#1%
		\char`\{$_1$%
	\else\ifx\visualtoks@cycle@egroup#1%
		\char`\}$_2$%
	\else
		\escapechar`\\%
		\edef\tempa{\string#1}%
		\escapechar\m@ne
		\edef\tempb{\string#1}%
		% #1 may be an \if... token, so we must use first/secondoftwo
		\ifx\tempa\tempb \expandafter\visualtoks@firstoftwo \else \expandafter\visualtoks@secondoftwo \fi
			{\string#1$_{%
				\ifcat $#13\fi
				\ifcat &#14\fi
				\ifcat###16\fi
				\ifcat ^#17\fi
				\ifcat _#18\fi
				\ifcat\space#110\fi
				\ifcat a#111\fi
				\ifcat !#112\fi
			}$}%
			{\visualtoksB{\string#1}}%
	\fi\fi\fi
}

\long\def\visualtoksB#1{%
	\hbox{%
		\vrule
		\vtop{%
			\vbox{%
				\hrule \kern\p@
				\hbox{\vphantom/\thinspace#1\thinspace}%
			}%
			\kern\p@ \hrule
		}%
		\vrule
	}%
}

\next
`.

% See the documentation for typeset examples and caveats.



\ifdefined\visualtoks
	\expandafter\endinput
\fi
\edef\next{\catcode`\noexpand\@=\the\catcode`\@\space}
\catcode`\@=11

% \visualtoks@cycle { tokens }
% Apply \visualtoksA to each token in tokens. Space tokens, left and right braces
% are replaced with \visualtoks@cycle@space, \visualtoks@cycle@bgroup, and
% \visualtoks@cycle@egroup respectively. Expandable.
\long\def\visualtoks@cycle#1{\visualtoks@cycleA#1{\visualtoks@cycle@nil}\visualtoks@cycleF}
\long\def\visualtoks@cycleA#1#{%
	\expandafter\visualtoks@cycleD
		\romannumeral\visualtoks@cycleB{}#1 \visualtoks@cycle@nil
	\visualtoks@cycleF \visualtoks@cycleE
}
\long\def\visualtoks@cycleB#1#2 #3\visualtoks@cycle@nil{%
	\if\relax\detokenize{#3}\relax \visualtoks@cycleC{#1#2}\fi
	\visualtoks@cycleB#1#2\visualtoks@cycle@space#3\visualtoks@cycle@nil
}
\long\def\visualtoks@cycleC#1\fi#2\visualtoks@cycle@nil{\fi\z@ #1\visualtoks@cycle@nil}
\long\def\visualtoks@cycleD#1{%
	\ifx\visualtoks@cycle@nil#1\visualtoks@cycleF\fi
	\visualtoksA{#1}\visualtoks@cycleD
}
\long\def\visualtoks@cycleE#1{%
	\ifx\visualtoks@cycle@nil#1\visualtoks@cycleF\fi
	\visualtoksA\visualtoks@cycle@bgroup
	\visualtoks@cycle{#1}%
	\visualtoksA\visualtoks@cycle@egroup
	\visualtoks@cycleA
}
\long\def\visualtoks@cycleF#1\visualtoks@cycleF{\fi}

\protected\def\visualtoks@cycle@bgroup{\noexpand\visualtoks@cycle@bgroup}
\protected\def\visualtoks@cycle@egroup{\noexpand\visualtoks@cycle@egroup}
\protected\def\visualtoks@cycle@space {\noexpand\visualtoks@cycle@space}

\protected\def\visualtoks@cycle@nil{\noexpand\visualtoks@cycle@nil}

\long\def\visualtoks@firstoftwo#1#2{#1}
\long\def\visualtoks@secondoftwo#1#2{#2}

\newdimen\visualtokskip \visualtokskip=1em

\long\def\visualtoks#1{%
	\begingroup
		\leavevmode
		\ifcsname ttfamily\endcsname \ttfamily \else \tt \fi
		\raggedright\frenchspacing
		\hskip-\visualtokskip \visualtoks@cycle{#1}%
	\endgroup
}
\long\def\visualtoksA#1{%
	\hskip\visualtokskip
	\ifx\visualtoks@cycle@space#1%
		\char`\ $_{10}$%
	\else\ifx\visualtoks@cycle@bgroup#1%
		\char`\{$_1$%
	\else\ifx\visualtoks@cycle@egroup#1%
		\char`\}$_2$%
	\else
		\escapechar`\\%
		\edef\tempa{\string#1}%
		\escapechar\m@ne
		\edef\tempb{\string#1}%
		% #1 may be an \if... token, so we must use first/secondoftwo
		\ifx\tempa\tempb \expandafter\visualtoks@firstoftwo \else \expandafter\visualtoks@secondoftwo \fi
			{\string#1$_{%
				\ifcat $#13\fi
				\ifcat &#14\fi
				\ifcat###16\fi
				\ifcat ^#17\fi
				\ifcat _#18\fi
				\ifcat\space#110\fi
				\ifcat a#111\fi
				\ifcat !#112\fi
			}$}%
			{\visualtoksB{\string#1}}%
	\fi\fi\fi
}

\long\def\visualtoksB#1{%
	\hbox{%
		\vrule
		\vtop{%
			\vbox{%
				\hrule \kern\p@
				\hbox{\vphantom/\thinspace#1\thinspace}%
			}%
			\kern\p@ \hrule
		}%
		\vrule
	}%
}

\next
