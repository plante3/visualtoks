\documentclass[11pt]{article}

\usepackage{geometry}

\usepackage{mlmodern}

\usepackage{enumitem}
\setlist[itemize]{align=parleft,left=0pt..30pt}

\parindent=0pt
\parskip=\medskipamount

\input visualtoks

\title{The \texttt{visualtoks} Package, version 1.0}
\author{plante}

\begin{document}

\maketitle

\noindent
In The \TeX book, Knuth demonstrates the concept of tokens with the following example:
\begin{quote}
	\medskip
	For example, if the normal conventions of plain \TeX are in force, the text
	`\verb|{\hskip 36 pt}|' is converted into a list of eight tokens:
	\[
		\hbox{\visualtokskip=1.25em \visualtoks{{\hskip 36 pt}}}
	\]
	The subscripts here are the category codes, as listed earlier: 1 for ``beginning of group,''
	12 for ``other character,'' and so on. The {\ttfamily \visualtoksB{\string\hskip}} doesn't
	get a subscript, because it represents a control sequence token instead of a character token.
	Notice that the space after \verb|\hskip| does not get into the token list, because it
	follows a control word. \hfill (p.\ 38){\parfillskip=0pt \par}
	\medskip
\end{quote}
The same style of token display is used several times in the \TeX book.
It would be useful to be able to generate the display automatically
for an arbitrary list of tokens, for pedagogical or debugging purposes.
This package provides the \verb|\visualtoks| command which does exactly that.

\subsection*{Usage}

Usage: \verb|\visualtoks{|$\,\langle\textit{token list}\rangle\,$\verb|}|.

This package may be used in plain \TeX\ or \LaTeX\ by \verb|\NeedsTeXFormat{LaTeX2e}
\ProvidesPackage{visualtoks}

\RequirePackage{tokmap}
\NeedsTeXFormat{LaTeX2e}
\ProvidesPackage{visualtoks}

\RequirePackage{tokmap}
\NeedsTeXFormat{LaTeX2e}
\ProvidesPackage{visualtoks}

\RequirePackage{tokmap}
\input{visualtoks}

\endinput

visualtoks.tex, version 1.1a

Copyright (C) 2025 plante
This package is released under the LaTeX Project Public License (LPPL) 1.3c.

visualtoks: typeset TeXbook-style visualisations of token lists.

Usage: `\visualtoks { <tokens> }`
This package may be used in plain TeX or LaTeX by `\input{visualtoks}`.

See the documentation for typeset examples and caveats.
See file `visualtoks.tex` for source code.


\endinput

visualtoks.tex, version 1.1a

Copyright (C) 2025 plante
This package is released under the LaTeX Project Public License (LPPL) 1.3c.

visualtoks: typeset TeXbook-style visualisations of token lists.

Usage: `\visualtoks { <tokens> }`
This package may be used in plain TeX or LaTeX by `\NeedsTeXFormat{LaTeX2e}
\ProvidesPackage{visualtoks}

\RequirePackage{tokmap}
\input{visualtoks}

\endinput

visualtoks.tex, version 1.1a

Copyright (C) 2025 plante
This package is released under the LaTeX Project Public License (LPPL) 1.3c.

visualtoks: typeset TeXbook-style visualisations of token lists.

Usage: `\visualtoks { <tokens> }`
This package may be used in plain TeX or LaTeX by `\input{visualtoks}`.

See the documentation for typeset examples and caveats.
See file `visualtoks.tex` for source code.
`.

See the documentation for typeset examples and caveats.
See file `visualtoks.tex` for source code.


\endinput

visualtoks.tex, version 1.1a

Copyright (C) 2025 plante
This package is released under the LaTeX Project Public License (LPPL) 1.3c.

visualtoks: typeset TeXbook-style visualisations of token lists.

Usage: `\visualtoks { <tokens> }`
This package may be used in plain TeX or LaTeX by `\NeedsTeXFormat{LaTeX2e}
\ProvidesPackage{visualtoks}

\RequirePackage{tokmap}
\NeedsTeXFormat{LaTeX2e}
\ProvidesPackage{visualtoks}

\RequirePackage{tokmap}
\input{visualtoks}

\endinput

visualtoks.tex, version 1.1a

Copyright (C) 2025 plante
This package is released under the LaTeX Project Public License (LPPL) 1.3c.

visualtoks: typeset TeXbook-style visualisations of token lists.

Usage: `\visualtoks { <tokens> }`
This package may be used in plain TeX or LaTeX by `\input{visualtoks}`.

See the documentation for typeset examples and caveats.
See file `visualtoks.tex` for source code.


\endinput

visualtoks.tex, version 1.1a

Copyright (C) 2025 plante
This package is released under the LaTeX Project Public License (LPPL) 1.3c.

visualtoks: typeset TeXbook-style visualisations of token lists.

Usage: `\visualtoks { <tokens> }`
This package may be used in plain TeX or LaTeX by `\NeedsTeXFormat{LaTeX2e}
\ProvidesPackage{visualtoks}

\RequirePackage{tokmap}
\input{visualtoks}

\endinput

visualtoks.tex, version 1.1a

Copyright (C) 2025 plante
This package is released under the LaTeX Project Public License (LPPL) 1.3c.

visualtoks: typeset TeXbook-style visualisations of token lists.

Usage: `\visualtoks { <tokens> }`
This package may be used in plain TeX or LaTeX by `\input{visualtoks}`.

See the documentation for typeset examples and caveats.
See file `visualtoks.tex` for source code.
`.

See the documentation for typeset examples and caveats.
See file `visualtoks.tex` for source code.
`.

See the documentation for typeset examples and caveats.
See file `visualtoks.tex` for source code.
|. The $\varepsilon$-\TeX\ extensions are required for the \verb|\detokenize| primitive.

$\langle\textit{token list}\rangle$ must be balanced with respect to explicit braces, and must not contain the token \verb|\visualtoks@cycle@nil|. It is assumed that \verb|{| is the only character with category code 1 (beginning of group).

The horizontal separation between displayed tokens may be configured by the \verb|\dimen| register \verb|\visualtokskip|. The default value is 1em.

\subsection*{Samples}

\begin{itemize}

\item \verb|\visualtoks{\def \macro {abc #1\egroup}}| gives
\[ \hbox{\visualtoks{\def \macro #1{abc #1\egroup}}}. \]

\item \verb|\visualtoks{$$\halign{&##\hfil\crcr}$$\par}| gives
\[ \hbox{\visualtoks{$$\halign{&##\hfil\crcr}$$\par}}. \]

\item Unbalanced \verb|\if...| tokens: \\
	\verb|\visualtoks{\ifnum\iffalse{\fi`} = 0\else}| gives
\[ \hbox{\visualtoks{\ifnum\iffalse{\fi`} = 0\else}}. \]

\makeatletter
\item To demonstrate how \TeX\ tokenizes consecutive spaces: \\
	\verb|\makeatletter |\verb*|\edef\temp{{    }{\@spaces}}| \\
	\verb|\expandafter\visualtoks\expandafter{\temp}| gives
\[ \hbox{\edef\temp{{    }{\@spaces}} \expandafter\visualtoks\expandafter{\temp}}. \]

\item To demonstrate the \verb|\lowercase| technique: \\
	\verb|\begingroup \lccode`a=`$ \lccode`?=`$ \lccode`#=`$ \lccode`|\verb*| =`$| \\
	\verb|\lowercase{\endgroup\def\temp{$a?## }}| \\
	\verb|\expandafter\demotokens\expandafter{\temp}| gives
\begingroup \lccode`a=`$ \lccode`?=`$ \lccode`#=`$ \lccode`\ =`$
\lowercase{\endgroup\def\temp{$a?## }}
\[ \hbox{\expandafter\visualtoks\expandafter{\temp}}. \]

\end{itemize}

\end{document}
